\chapter{针对中医医案症状术语实体的识别}
\section{任务背景}
中医典籍和医案是中国传统医学宝贵的知识财富。
中医医案经历了长时间的积累,记载了古今众多医者的实证经验和临床诊疗规律。
这些医案中存在大量的症状信息,这些信息对医者辩证论治、察明证素,进而对证施治起着至关重要的作用;
同时,症状、治法、处方、剂量与预后疗效之间存在的内在联系,也能够反映中医对疾病的诊疗机理。
若要系统地研究这些诊疗机理,对医案进行结构化处理,挖掘医案文本,寻找其中蕴含的关联规则是一系列必要的任务。
在医案结构化处理任务中,首要的就是将中医医案文本中对症状具体的描述和长期以来形成的普遍使用的中医症状术语提取出来。

目前,针对中医医案文本中症状术语的识别存在几个难点。
首先,中医医案的结构化程度低。
具体表现为不同年代、不同地域的不同医家个体,其记录医案的结构和习惯均存在不同,导致无法按统一标准进行结构化。
其次,就医案内容而言,由于存在大量医案年代较为久远,其语言特点与现代汉语也存在较大差异,多为文言文或半文半白,识别难度较大。
最后,对于需要提取的目标症状描述或症状术语,缺乏统一的规范。
至今,中医临床常见症状术语尚未有国家标准,中医医案中口语化、症状实体包含、互指现象也十分普遍,加之症状描述本身形式灵活多变,这对识别症状的正确性、完整性提出了更高的要求。

识别中医医案文本中的症状描述或症状术语这一任务实际上就是从医案文本中进行症状实体识别,可以将其视为命名实体识别任务的一种。
目前对于中医医案症状识别主要使用CRF方法。
文献[]从命名实体识别的方法出发,对比了CRF与支持向量机、最大熵模型等常见的命名实体识别方法,指出了CRF在该任务上的有效性,但同时也存在依赖分词和词性标注效果、对超过五个字符的较长实体识别效果较差等局限性。
文献[]对比了CRF、隐马尔科夫模型和最大熵隐马模型在特定病种的标准病历数据集上对症状、疾病和诱因的识别效果。
文献[]-[]均采用了CRF方法针对中医领域的文本语料进行命名实体识别,以进行医案结构化、症状等中医术语提取等工作。

本章针对CRF方法特征选择相对复杂,人工标注特征代价相对较高,对语料结构化、标准化程度依赖较大等局限性,提出使用结合神经网络的序列标注框架,
在以文言文本为主的近代中医医案典籍上进行中医症状术语识别,并根据症状组成要素,添加简单的字符级别特征,在增加较少人工标注成本的前提下,获得更佳的症状识别效果。
本章将使用不同的LSTM结构进行实验,并对比分析不同结构对实验结果的影响,考察章节\ref{chap:3}介绍的标注模型在以文言文为主的语料上的效果。

\section{实验数据}
本章使用的数据集来自中医典籍《全国名医验案类方》,作者为近代名医何廉臣,共计492例医案。
原医案语料已具有一定的结构特征,单个医案中划分了病因、证候、诊断、疗法等不同项目,但也存在包括医者、病者等无关信息,本节只选取其中证候、诊断、疗法、效果及作为医案评述的“廉按”部分内容作为原始语料,共2959条语料。
由于典籍中的医案是医者行医记录,虽然在医者已按照既定结构记载,但这些项目中仍存在大量描述和推断,不能完全作为症状实体。
且记录医案存在一定随意性,症状实体在证候、诊断、疗法、效果和医案评述中均可能出现,故对所有语料参考《中医临床症状术语规范》中六大类37小类共2069条规范症状进行标注。
考虑到原始语料年代较为久远,部分症状描述可能与现代汉语存在差异,故酌情对语料中频繁出现的部分症状进行了补充标注。

\begin{figure}[H]
    \centering
    \includegraphics[width=0.8\textwidth]{Medical_case.png}
    \caption{医案原始文本}
    \label{fig:medical_case}
\end{figure}

需要说明的是,文献[]和[]的工作已经表明,对于中医医案尤其是使用古代汉语的中医医案语料,对其进行分词效果并不好,因为现有的分词工具多倾向于将无法识别的词语进行单字切分,若以词为粒度进行识别,分词会对结果产生负面影响。
因此本节利用字符级别特征进行序列标注,对所有语料进行单字切分,并按照BIO模式进行标注,即症状实体起始字标签为“B”,症状实体非起始字为“I”,非症状实体字为“O”。
由于实验语料规模较小,故实验中将随机抽取其中30例医案文本作为测试语料,剩余语料作为训练数据,其中训练集与验证集比例为9:1。

最后,为了得到相对较好的结果,实验中还使用Gensim工具利用全部中医医案语料作为训练数据,获得了预训练的字嵌入作为基本特征,字嵌入维度设置为100。
本章基于是否加入症状字特征和多种LSTM cell实现进行了对比实验。

\section{增加症状字特征的症状实体识别}
考虑到中医症状实体本身还具有很多明显的特征,以这些特征作为锚点,能够快速准确地定位一些出现比较频繁的症状。
但同时也存在小部分负例,应以文本序列出现的上下文信息判断是否是症状。

为了考察这些特征对症状识别结果的影响,本节进行了如表\ref{tab:symptom_cluster}所示的特征标签分类。
对于文本序列中的单字,模型使用预训练的字嵌入作为基本特征;额外的症状术语单字,将采用随机初始化获得其表示其标签类型的嵌入向量,与字嵌入连接作为该字符的表示。
表\ref{tab:symptom_cluster}所示特征,仅为对标准化的症状词典进行简单的字符频率统计并按照字符表意类型进行分类得到的常见额外字特征。
\begin{table}[H]
    \centering
    \footnotesize
    \setlength{\tabcolsep}{4pt}
    \renewcommand{\arraystretch}{1.2}
    \begin{tabular}{cp{5cm}c}
        \toprule
        特征种类 & 部分特征关键字 & 特征标签\\
        \midrule
        自感特征 & 疼、痛、肿、胀、酸、麻 & SF(self feeling)\\
        \midrule
        身体部位 & 脉、左、右、寸、关、尺、舌、苔、手、身、口、目、眼、耳、足、胸、额、头、脘、腹、唇、齿、胃、面、颊、肢 & BP(body parts)\\
        \midrule
        颜色形态 & 白、赤、红、滑、腻、溏、稀、干、燥、朱、黑、青、紫、绛、焦、黄 & CS(color shape)\\
        \midrule
        程度描述 & 稍、微、略、甚、强、大、多、少、无、极、弱 & AD(adverbs of degree) \\
        \midrule
        排泄物名词 & 便、溲、痰、汗 & EC(ecreta)\\
        \bottomrule
    \end{tabular}
    \caption{症状字特征分类}
    \label{tab:symptom_cluster}
\end{table}

上述特征均是字符级别。
为了不增加训练语料的人工标注成本,可以在训练语料中进行简单的单字替换获得,作为单独的特征列输入模型。
最终的标注结果示例如图5所示。
\begin{figure}[H]
    \centering
    \includegraphics[width=0.5\textwidth]{Character-Level-Feature}
    \caption{增加字特征的医案标注文本示例}
    \label{fig:character_level_feature}
\end{figure}

本节将单字特征和预训练的字嵌入进行连接获得的新的字向量表示作为模型输入,这些额外特征将作为字嵌入的扩展,为模型提取特征从而进行症状术语识别提供更具体有效的信息。
基于症状要素的字特征嵌入维度设置为30。

\section{整体架构}
根据章节\ref{chap:3}的介绍,我们基于双向LSTM-CRF框架构建了序列标注系统。

\begin{figure}[H]
    \centering
    \includegraphics[width=\linewidth]{TCM-NER}
    \caption{双向LSTM-CRF中医医案术语标注框架结构}
    \label{fig:TCM_NER}
\end{figure}
如图\ref{fig:TCM_NER}所示,输入特征为预训练词向量和字特征向量的连接。
LSTM层获取序列上下文特征,CRF层通过以上下文特征作为状态特征,同时训练标签转移矩阵参数,获得转移特征。
最后通过Viterbi算法在可行标签序列中,获得评分最高的序列作为最终预测序列输出。

本节使用隐藏层单元的数量为64,训练语料句子长度最大不超过100个字符。
为防止过拟合,建立模型时对LSTM的输入层和输出层进行随机dropout,保留概率均设置为0.8。
同时,全连接层L2正则化参数为0.01,全局学习率为0.002。

\section{评价指标}
\label{sec:tcm-pfr}
前面提到,中医症状术语既没有国家标准可以遵照,也没有公认的一套界定方案,因此其评价标准是按照一般命名实体整体对待,还是以字粒度为主,尚有待讨论。
且目前已有的在中医文本上进行命名实体识别的工作,多采用字符级别特征。参考文献\citepns{万静2016基于条件随机场的医药领域症状信息抽取},我们选择以标签为对象的评价标准。实验结果以标签为粒度,对BI标签分别按表\ref{tab:confusion}统计准确率(Precision)、召回率(Recall)和F1值。

\begin{table}
    \centering
    \begin{tabular}{ccc}
        \toprule
        \diagbox{真实结果}{预测结果} & 正类 & 负类\\
        \midrule
        正类 & True Positive (TP) & False Negative (FN)\\
        负类 & False Positive (FP) & True Negative (TN)\\
        \bottomrule
    \end{tabular}
    \caption{混淆矩阵}
    \label{tab:confusion}
\end{table}

分别统计得到B、I的$TP$、$FP$、$FN$后,则两种标签的准确率$Precision$、召回率$Recall$和F值可以由式\ref{eq:preision}-\ref{eq:f-measure}计算。

\begin{align}
    Precision &= \frac{TP}{TP+FP} \label{eq:preision}\\
    Recall &= \frac{TP}{TP+FN}\\
    F_\beta &= \frac{(1 + \beta^2)\times Precision \times Recall}{(\beta^2\times Precision) + Recall}\label{eq:f-measure}
\end{align}

其中$\beta$是正实数,表示$Precision$相对$Recall$的重要性。本节实验取$\beta = 1$。

\section{实验结果与分析}
本章分别对基本LSTM单元、带peep-hole机制的LSTM单元、GRU和CIFG四种不同的LSTM实现进行了对比实验,并在此基础上增加了症状字特征,表\ref{tab:tab1}展示了实验的结果。
\begin{table}[H]
    \centering
    \begin{tabular}{cccccccccc}
        \toprule
            \multirow{2}{*}{Cell} &\multicolumn{3}{c}{Overall} &\multicolumn{3}{c}{B} &\multicolumn{3}{c}{I}\\
            \cmidrule(lr){2-4} \cmidrule(lr){5-7} \cmidrule(lr){8-10}
            & F1 & R & P & F1 & R & P & F1 & R & P\\
        \midrule
            LSTM & *0.724 & 0.621 & 0.866 & 0.736 & 0.635 & 0.874 & 0.718 & 0.600 & 0.894\\
            (baseline) & 0.756 & 0.675 & 0.860 & 0.763 & 0.685 & 0.862 & 0.753 & 0.653 & 0.889\\
            LSTM & *0.748 & 0.691 & 0.815 & 0.746 & 0.689 & 0.812 & 0.748 & 0.670 & 0.848\\
            (peep-hole) & 0.763 & 0.699 & 0.839 & 0.748 & 0.685 & 0.824 & 0.769 & 0.687 & 0.873\\
            \multirow{2}{2cm}{GRU} & *0.754 & 0.717 & 0.796 & 0.738 & 0.689 & 0.794 & 0.761 & 0.698 & 0.835\\
                                 & 0.762 & 0.708 & 0.825 & 0.758 & 0.707 & 0.816 & 0.764 & 0.687 & 0.861\\
            \multirow{2}{2cm}{CIFG} & *0.742 & 0.664 & 0.839 & 0.749 & 0.662 & 0.863 & 0.738 & 0.640 & 0.872\\
                                  & 0.780 & 0.735 & 0.832 & 0.762 & 0.707 & 0.826 & 0.788 & 0.725 & 0.863\\
        \bottomrule
    \end{tabular}
    \caption{加入症状字特征识别效果对比}
    \label{tab:tab1}
\end{table}
其中,标$\*$号行为未加入症状字特征的实验数据。
实验结果表明,基于LSTM-CRF的序列标注模型在中医医案症状识别任务中F1值最好达到了0.78。

在对中医医案进行序列标注的任务中,提高召回率是一个难点,这与文献[]在相似语料上的实验得出的结论一致。
在未加入症状字特征的条件下,得益于相对简单的结构和较少的待训练参数,GRU取得了相对较好的识别效果。
增加症状字特征后,可以看出不同的LSTM实现在标签识别的效果上均有不同程度的提高,说明常见症状字特征对识别结果有一定的积极影响。

其中,典型症状如“<++>”、“<++>”等具有较为明显的特征,其上下文也相对简单,故识别效果较好;经过比对发现,未识别正确的症状包括“<++>”、“<++>”、“<++>”等,其症状形式较为多变,标准难以统一,这也造成了许多类似症状识别的困难。

\begin{table}[H]
    \centering
    \begin{tabular}{ccccccc}
        \toprule
            \multirow{2}{*}{Cell} & \multicolumn{3}{c}{B} & \multicolumn{3}{c}{I}\\
            \cmidrule(lr){2-4} \cmidrule(lr){5-7}
            & F1 & R & P & F1 & R & P\\
        \midrule
        {LSTM}{(baseline)} & 0.744&0.612&0.950&0.744&0.600&0.979\\
        {LSTM}{(peep-hole)} & 0.769&0.645&0.952&0.755&0.625&0.952\\
        {GRU} & 0.693&0.548&0.944&0.668&0.518&0.943\\
        {CIFG} & 0.754&0.645&0.909&0.770&0.643&0.962\\
        \bottomrule
    \end{tabular}
    \caption{LSTM-CRF对长症状术语的识别效果}
    \label{tab:tab2}
\end{table}
另外,CIFG在加入症状字特征后,各标签的召回率和精确率都得到了提高。
也可以看出,与基线实验相比,LSTM(with peep-hole)、GRU和CIFG三种带有peep-hole实现的单元结构对增加的特征更加敏感,特征对提升它们在症状实体识别任务上的表现具有更明显的作用。
同时,使用LSTM-CRF序列标注模型相比单纯的CRF能够更好地识别较长的症状术语,改善了CRF对长症状术语不能很好召回的情况[]。
各方法对测试语料中五字以上长症状术语的识别效果如表\ref{tab:tab2}所示。

\section{本章小结}
本章在基于LSTM-CRF序列标注模型的基础上,针对中医症状特征添加了部分字符级别的要素特征,在小规模以古汉语为主的中医医案语料上的实验表明,该模型泛化能力更强,且对较长实体的识别效果更好。
同时,由于中医症状种类繁多,加之古汉语精干灵活的语言特点,该方法也存在召回率较低的缺点。
因此后续工作将围绕以下几点展开:第一,比较多种不同的标签模式对症状术语识别的效果影响;第二,由于部分症状描述分散于多个短语中,并且存在大量比较(如“六脉俱缓,左关尤甚”)、持续(如“继则全体大热,昼夜不休”)等特征,寻找有效方法利用这些特征也至关重要;最后还需考虑参照症状字典对不标准的症状描述进行聚类,识别时按照标准症状建立自然语言序列与症状的映射关系,便于以后对古医案的结构化。



