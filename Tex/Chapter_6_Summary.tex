\chapter{总结与展望}
\section{全文总结}
命名实体识别虽然是自然语言处理中的较为基础的任务,但在多个应用领域扮演了非常重要的角色。
众多自然语言处理系统都依赖序列标注的结果,其中多数都在模型、方法和系统等不同层面上整合了命名实体识别系统做为模块,如在事件提取\citep{nguyen2016joint}和关系提取\citep{miwa2016end}中,命名实体作为上层模块识别对象的一部分,其识别的结果对整个系统的性能起着至关重要的作用。

本文从早期基于统计机器学习的命名实体识别方法作为切入点,探讨了在一定的概率假设前提下进行命名实体识别的主要思路和常见方法。
其中的一些方法时至今日仍是在生产环境中最高效、最常用的方法。
需要承认的一点是,在现今基于神经网络和深度学习的方法如此流行且发展迅速的情形下,
完善用户字典仍然是提高一般语料上命名实体识别效果最廉价、最有效的手段。
尽管如此,其局限性也随着数据规模的改变、使用维护成本等问题而日益放大。
本文在原理和应用场景上对这些方法进行了分析比较,并讨论了其中一些方法存在的问题和缺陷。

为了能够适应新的需求,本文详细讨论了基于神经网络与深度学习的命名实体识别方法。
在这一章中,我们详细讨论了词向量的基本原理与获取方式,并对比了四种不同的LSTM cell实现。最后整体介绍了双向LSTM-CRF序列标注框架,并解释了CRF层在框架中的作用。同时,针对深度学习中模型训练的常见问题,本文列举了几种解决方法以应用到后续的实践当中。

基于前文的理论,本文还进行了两方面的研究工作:
\begin{enumerate}[\indent(1)]
    \item 文言文作为汉语言的一部分,命名实体识别在包括中医医案在内的非现代汉语文本上也具有广阔的应用空间。本文设计实现了前文所述的双向LSTM-CRF序列标注框架,并针对中医医案中症状术语实体的组成成分制订了字符级别特征,这些特征基本不需要花费额外的标注成本,但对症状术语的识别效果起到了一定的提升作用。
    \item 针对公共领域语料,本文讨论了在基于神经网络的方法下以字、词向量作为模型输入的缺陷,并提出了整合字词向量作为预训练嵌入和训练、测试语料处理的方法。实验证明,该方法能够在一定程度上克服字向量导致序列过长的问题,也能够较好地改善词向量识别效果差的情况,在不过多牺牲识别精度的同时,显著减少模型训练时间。
\end{enumerate}

\section{工作展望}
本文的研究工作尚存在一些不足。
\begin{enumerate}[\indent(1)]
    \item 本文提出的方法仍然在一定程度上依赖其他序列标注任务的结果,这种依赖虽然不需要作为特征输入模型,但在对预测语料预处理的过程中隐含传递给了嵌入的选择上
    \item 相比成熟的传统方法搭配较为完善的字典与领域知识系统,本文的实验结果仍存在一定差距
    \item 本文在模型层面上对命名实体识别还存在提升空间
\end{enumerate}

本文下一步的研究方向主要围绕两个方面展开:
\begin{enumerate}[\indent(1)]
    \item 针对模型的改进,下一步可进行的工作包括:
        \begin{enumerate}[\indent(a)]
            \item 在前人引入序列到序列的机器翻译模型进行命名实体识别的基础上\citep{隋臣2017基于深度学习的中文命名实体识别研究},进一步考察注意力机制(Attention)在中文命名实体识别上的作用原理,更好地融合字词向量特征
            \item 针对LSTM到CRF的输出层进行改进,尝试建立语言模型对LSTM的输出进行干预,以使CRF能够在更贴近真实语言环境的情况下进行标签序列预测
        \end{enumerate}
    \item 针对模型的应用,下一步可进行的工作包括:设计实现可用的统一命名实体识别系统,同步识别中英文命名实体以用于字幕、双语文献、机器翻译平台等场景下。
\end{enumerate}
