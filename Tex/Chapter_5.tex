\chapter{结合字词向量的命名实体识别}
\section{基于字词向量结合的特征嵌入与训练}
\subsection{嵌入粒度对命名实体识别的影响}
现有的命名实体识别方法一般基于两个思路,一是以不定长的词为粒度,依赖词性、上下文等特征,对特定的一些词序列进行标注或聚合(分块),最终得到与原文对应的标签序列;二是以字为粒度,依赖字本身的特征进行字符序列的标注。

思路一常见于以传统的基于统计的方法进行的命名实体识别中。
但该方法在实际应用中存在一定缺陷。
首先,基于词粒度的命名实体识别较为依赖分词的结果、词性等先验特征,分词结果的好坏与词性标注的结果将对命名实体识别的结果产生较大的影响;
其次,分词这一任务本身是识别语料中单独成词的序列,而命名实体中存在大量多个词组合而成复合实体,虽然分词的目的并不在于确定命名实体的词边界,但错误的分词结果将会消除命名实体的词边界,在此基础上识别的命名实体必然是错误的;
最后,即便分词结果相对可靠,但由于命名实体本身的结构是多变的,相同的词并不一定总是命名实体,而在训练语料中未被单独分出的词在测试语料中是命名实体,或者测试语料中是命名实体的序列根本未在训练语料中出现过,并且其也并不由训练预料中的某个或某些词组成,那么这样的命名实体就很难被识别出来。

\begin{table}[H]
    \centering
    \caption{基于词的识别部分伪正类和伪负类}
    \begin{tabular}{cccc}
        \toprule
        \multicolumn{2}{c}{False positive} & \multicolumn{2}{c}{False negative}\\
        \midrule
        错识别标签 & 字串 & 未识别标签 & 字串\\
        \cmidrule(lr){1-2}\cmidrule(lr){3-4}
        \multirow{3}{*}{PERSON} & 钟声 & \multirow{3}{*}{PERSON} & 安南 \\
        & 唐诗 & & 简 \\
        & 宋词 & & 老舍\\
        \cmidrule(lr){1-2}\cmidrule(lr){3-4}
        \multirow{2}{*}{LOC} & 苏东坡 & \multirow{2}{*}{LOC}& 梁园\\
        & 关牧村 & & 孟买 \\
        \cmidrule(lr){1-2}\cmidrule(lr){3-4}
        \multirow{2}{*}{ORG} & 复印社 & \multirow{2}{*}{ORG}& 国家银行\\
        & 台湾公司 & & 联合国\\
        \bottomrule
    \end{tabular}
    \label{tab:fp_fn_word}
\end{table}

其原因在于,在进行命名实体识别时,忽略了构成词的字的特征,如“梁孝王刘武……”,由于该序列中存在四个常见的人名姓氏,因此不论是直接分词,还是以姓氏字典作为触发字的识别,在这种情况下都会受到严重干扰,导致最终结果不正确。

总之,本文认为基于词的命名实体识别方法仅在语料规律相对明显,内容相对较为简单,且具备大规模的、成熟的人名、地名和机构名词典或特殊领域知识的情况下,具有很好的效果。
在处理未登录词问题上,还存在较大缺陷。而这里我们尚且仅考虑人名、地名和机构名,并未考虑其他特殊专名、网络新词与领域新词等其他命名实体,因为就命名实体识别而言,存在周期相对较短的部分新词并不作为该任务的关注重点。

\begin{table}[H]
    \centering
    \caption{基于词的识别部分伪正类和伪负类}
    \begin{tabular}{cccc}
        \toprule
        \multicolumn{2}{c}{False positive} & \multicolumn{2}{c}{False negative}\\
        \midrule
        错识别标签 & 字串 & 未识别标签 & 字串\\
        \cmidrule(lr){1-2}\cmidrule(lr){3-4}
        \multirow{3}{*}{PERSON} & 钟声 & \multirow{3}{*}{PERSON} & 安南 \\
        & 唐诗 & & 简 \\
        & 宋词 & & 老舍\\
        \cmidrule(lr){1-2}\cmidrule(lr){3-4}
        \multirow{2}{*}{LOC} & 苏东坡 & \multirow{2}{*}{LOC}& 梁园\\
        & 关牧村 & & 孟买 \\
        \cmidrule(lr){1-2}\cmidrule(lr){3-4}
        \multirow{2}{*}{ORG} & 复印社 & \multirow{2}{*}{ORG}& 国家银行\\
        & 台湾公司 & 联合国& \\
        \bottomrule
    \end{tabular}
    \label{tab:fp_fn_word}
\end{table}

思路二是常见于基于神经网络的方法。考虑到以词为
虽然以字为基本元素进行命名实体识别能够取得不错的效果,但其也存在一些问题。
首先,以字符为粒度对语料进行切分,会产生一些过长的字符序列。尽管LSTM在理论上能够接受任意长度的输入序列,但过长的序列会产生层数非常深的LSTM网络,这将使得网络参数量大大增加,训练速度相应降低。
其次,尽管LSTM具有一定的获取长距离上下文语义信息的能力,但若不对时间步的最大值加以限制,BPTT算法在进行误差的反向传播时,还是会面临严重的梯度消失问题,使得模型训练的结果变差。
最后,在神经网络模型中,字的表示来源于根据训练语料获得的字向量或通过大规模未标注语料预训练的字向量。
其信息往往不能完全覆盖预测环境下的语义,部分非命名实的常用字可能也经常出现在命名实体中,而部分非命名实体中的罕用字由于缺乏训练语料的语义信息,往往多被判别为命名实体。

对于序列过长及其导致的梯度消失问题,可行的解决办法包括通过去除停用字、词控制序列长度,制定截断策略在设定的阈值前后切分长序列和对原序列进行采样等方法。本文中所使用训练语料大多不超过250字符,故在以字为基本单位时,序列长度设定不超过250。

而对于同字异义、异字同意等问题,往往来源于歧义。这一部分目前较难从根本上得到解决。

基于上述分析,本文提出一种在模型训练中结合字词向量的方法。该方法能够在降低分词效果对命名实体识别准确率的影响的同时,尽量减少无关的非命名实体的干扰。
\subsection{字向量结合的嵌入和训练}
结合字词向量预训练词嵌入与模型训练的基本思路是,以词性为辅助依据,对语料进行切分。

总体流程可如图所示。
\begin{figure}[htpb]
    \centering
    \caption{Name}
    \includegraphics[width=\linewidth]{TCM-NER}
    \label{fig:exp_frame}
\end{figure}

\section{实验设计与数据预处理}
本章实验包括三个部分。首先,先采用第四章介绍的双向LSTM-CRF序列标注框架,分别建立使用字向量、词向量作为训练数据的的基线模型,并在这一部分考察使用不同LSTM实现对公开领域语料命名实体识别的效果,对比不同模型对词性特征的依赖程度。

其次,选取在基线实验部分表现较好的实现作为进一步的实验框架,增加结合字词向量作为训练数据的是实验组,调整模型参数,采用五折交叉验证获取多组实验数据,求得准确率、召回率

最后,在减少词性特征的条件下重复实验,对比分析不同模型在不依赖词性的情况下进行命名实体识别的效果。


实验数据的准备包含两部分:字词向量嵌入和训练语料预处理。

字词向量的嵌入采用中文维基百科语料库,总大小约1.5G。使用Word2Vector的实现之一Gensim进行字词向量的训练。
其中,对字进行训练是针对命名实体,可能包含较多不常用字,因此字向量的获取采用Hierachical Softmax模型,上下文窗口设置为5,采样率为$10^-4$,最小词频设置为1,获取所有字的嵌入。
对词向量的训练针对常见词,模型采用Negative Sampling,最小词频设置为5,过滤掉罕见词。
最终获取的词向量为300维,其中字向量22884个,词向量
%TODO

训练语料来源于人民日报,包含1998年1至6月经人工标注的语料。原始语料格式如下:

\begin{verbatim}
本报/r  哈尔滨/ns  1月/t  1日/t  电/n  记者/n  董/nr  伟/nr
报道/v  :/w  顶/v  着/u  寒风/n  ,/w  黑龙江省/ns  各级/r
领导/n  组成/v  若干/m  慰问/vn  小组/n  ,/w  在/p  节日/n
期间/f  深入/v  到/v  厂矿/j  车间/n  ,/w  为/p  困难/a  职工/n
送/v  去/v  党/n  的/u  温暖/an  。/w
\end{verbatim}

预处理过程中根据已经标注好的词性进行如下处理:
\begin{enumerate}
    \item 去除特殊符号、空标注等非法字符
    \item 去除l、i等与本文任务无关的组合词标签
    \item 合并组合命名实体,根据词性标签nr, ns, nt分别标注命名实体及其组成元素。标注模式如表x所示
    \item 合并所有名词类词性标签,去除人名、地名、机构名的二级类别标签,全部用一级类别n取代
    \item 分别生成字、词和字词结合的训练语料与测试语料
    \item 去除所有词性标签,作为对比实验语料
\end{enumerate}

对人民日报原始语料进行实体标签标注时采用BIO模式,标签设置如下表所示。

\begin{table}[H]
    \centering
    \caption{标注模式}
    \begin{tabular}{ll}
        \toprule
            BIO标签集\\
        \midrule
        B-PER & 人名起始及单字(词)人名 \\
        I-PER & 人名中间字词 \\
        B-LOC & 地名起始及单词地名 \\
        I-LOC & 地名中间字词 \\
        B-ORG & 机构名起始及单词机构名 \\
        I-ORG & 机构名中间字词 \\
        O & 非命名实体\\
        \bottomrule
    \end{tabular}
\end{table}

实验平台基于Ubuntu 16.04LTS,CPU为Intel I5 8600K CPU,GPU 为nVidia GeForce GTX 1070Ti,内存为16G。双向LSTM-CRF序列标注框架的实现基于TensorFlow-GPU 1.6.0,Python版本为3.5。
所有实验均使用GPU加速,CUDA版本为8.0, cuDNN版本为7.0。

模型的超参数设置如下:
\begin{table}[H]
    \centering
    \caption{参数列表}
    \begin{tabular}{ll}
        \toprule
        \multicolumn{2}{c}{模型超参数设置} \\
        \midrule
        batch\_size & 512(字嵌入)/800(词嵌入) \\
        隐藏层单元数量 & 256\\
        全局drop\_out\_rate & 0.5 \\
        l2正则化参数 & 0.01 \\
        初始学习率 & 0.002 \\
        早停等待epoch数 & 3\\
        序列最大长度 & 250(字嵌入)/120(词嵌入)\\
        \bottomrule
    \end{tabular}
\end{table}
其中,batch\_size由于显存限制,最大设置为表中数值;实际选择结果最优的参数重复实验。训练集与验证集的比例为9:1。

\section{实验结果与分析}
实验结果的评价指标参考第四章。本节以整个命名实体识别结果作为评判对象。模型识别结果与实际结果完全一致则可归为正类,否则即为负类。

首先,在有词性作为额外特征的情况下,基于字向量的命名实体识别在不同LSTM实现的模型表现如下表。
\begin{table}[H]
    \centering
    \caption{实验一:字嵌入+不同LSTM实现的命名实体识别结果}
    \begin{tabular}{cccccccccc}
        \toprule
            \multirow{2}{*}{Cell} &\multicolumn{3}{c}{Overall} &\multicolumn{3}{c}{B} &\multicolumn{3}{c}{I}\\
            \cmidrule(lr){2-4} \cmidrule(lr){5-7} \cmidrule(lr){8-10}
            & F1 & R & P & F1 & R & P & F1 & R & P\\
        \midrule
            LSTM & *0.724 & 0.621 & 0.866 & 0.736 & 0.635 & 0.874 & 0.718 & 0.600 & 0.894\\
            (baseline) & 0.756 & 0.675 & 0.860 & 0.763 & 0.685 & 0.862 & 0.753 & 0.653 & 0.889\\
            LSTM & *0.748 & 0.691 & 0.815 & 0.746 & 0.689 & 0.812 & 0.748 & 0.670 & 0.848\\
            (peep-hole) & 0.763 & 0.699 & 0.839 & 0.748 & 0.685 & 0.824 & 0.769 & 0.687 & 0.873\\
            \multirow{2}{2cm}{GRU} & *0.754 & 0.717 & 0.796 & 0.738 & 0.689 & 0.794 & 0.761 & 0.698 & 0.835\\
                                 & 0.762 & 0.708 & 0.825 & 0.758 & 0.707 & 0.816 & 0.764 & 0.687 & 0.861\\
            \multirow{2}{2cm}{CIFG} & *0.742 & 0.664 & 0.839 & 0.749 & 0.662 & 0.863 & 0.738 & 0.640 & 0.872\\
                                  & 0.780 & 0.735 & 0.832 & 0.762 & 0.707 & 0.826 & 0.788 & 0.725 & 0.863\\
        \bottomrule
    \end{tabular}
    \label{tab:tab1}
\end{table}

基于字向量的命名实体识别在不同LSTM实现的模型表现如下表。
\begin{table}[H]
    \centering
    \caption{实验二:词嵌入+不同LSTM实现的命名实体识别结果}
    \begin{tabular}{cccccccccc}
        \toprule
            \multirow{2}{*}{Cell} &\multicolumn{3}{c}{Overall} &\multicolumn{3}{c}{B} &\multicolumn{3}{c}{I}\\
            \cmidrule(lr){2-4} \cmidrule(lr){5-7} \cmidrule(lr){8-10}
            & F1 & R & P & F1 & R & P & F1 & R & P\\
        \midrule
            LSTM & *0.724 & 0.621 & 0.866 & 0.736 & 0.635 & 0.874 & 0.718 & 0.600 & 0.894\\
            (baseline) & 0.756 & 0.675 & 0.860 & 0.763 & 0.685 & 0.862 & 0.753 & 0.653 & 0.889\\
            LSTM & *0.748 & 0.691 & 0.815 & 0.746 & 0.689 & 0.812 & 0.748 & 0.670 & 0.848\\
            (peep-hole) & 0.763 & 0.699 & 0.839 & 0.748 & 0.685 & 0.824 & 0.769 & 0.687 & 0.873\\
            \multirow{2}{2cm}{GRU} & *0.754 & 0.717 & 0.796 & 0.738 & 0.689 & 0.794 & 0.761 & 0.698 & 0.835\\
                                 & 0.762 & 0.708 & 0.825 & 0.758 & 0.707 & 0.816 & 0.764 & 0.687 & 0.861\\
            \multirow{2}{2cm}{CIFG} & *0.742 & 0.664 & 0.839 & 0.749 & 0.662 & 0.863 & 0.738 & 0.640 & 0.872\\
                                  & 0.780 & 0.735 & 0.832 & 0.762 & 0.707 & 0.826 & 0.788 & 0.725 & 0.863\\
        \bottomrule
    \end{tabular}
    \label{tab:tab1}
\end{table}
(一顿分析)

在去除词性作为额外特征后,训练语料仅以嵌入特征作为输入,基于字、词向量的命名实体识别在不同LSTM实现的模型表现分别如表x和表x所示。
\begin{table}[H]
    \centering
    \caption{实验三:词嵌入+不同LSTM实现的命名实体识别结果}
    \begin{tabular}{cccccccccc}
        \toprule
            \multirow{2}{*}{Cell} &\multicolumn{3}{c}{Overall} &\multicolumn{3}{c}{B} &\multicolumn{3}{c}{I}\\
            \cmidrule(lr){2-4} \cmidrule(lr){5-7} \cmidrule(lr){8-10}
            & F1 & R & P & F1 & R & P & F1 & R & P\\
        \midrule
            LSTM & *0.724 & 0.621 & 0.866 & 0.736 & 0.635 & 0.874 & 0.718 & 0.600 & 0.894\\
            (baseline) & 0.756 & 0.675 & 0.860 & 0.763 & 0.685 & 0.862 & 0.753 & 0.653 & 0.889\\
            LSTM & *0.748 & 0.691 & 0.815 & 0.746 & 0.689 & 0.812 & 0.748 & 0.670 & 0.848\\
            (peep-hole) & 0.763 & 0.699 & 0.839 & 0.748 & 0.685 & 0.824 & 0.769 & 0.687 & 0.873\\
            \multirow{2}{2cm}{GRU} & *0.754 & 0.717 & 0.796 & 0.738 & 0.689 & 0.794 & 0.761 & 0.698 & 0.835\\
                                 & 0.762 & 0.708 & 0.825 & 0.758 & 0.707 & 0.816 & 0.764 & 0.687 & 0.861\\
            \multirow{2}{2cm}{CIFG} & *0.742 & 0.664 & 0.839 & 0.749 & 0.662 & 0.863 & 0.738 & 0.640 & 0.872\\
                                  & 0.780 & 0.735 & 0.832 & 0.762 & 0.707 & 0.826 & 0.788 & 0.725 & 0.863\\
        \bottomrule
    \end{tabular}
    \label{tab:tab1}
\end{table}

\begin{table}[H]
    \centering
    \caption{实验四:词嵌入+不同LSTM实现的命名实体识别结果}
    \begin{tabular}{cccccccccc}
        \toprule
            \multirow{2}{*}{Cell} &\multicolumn{3}{c}{Overall} &\multicolumn{3}{c}{B} &\multicolumn{3}{c}{I}\\
            \cmidrule(lr){2-4} \cmidrule(lr){5-7} \cmidrule(lr){8-10}
            & F1 & R & P & F1 & R & P & F1 & R & P\\
        \midrule
            LSTM & *0.724 & 0.621 & 0.866 & 0.736 & 0.635 & 0.874 & 0.718 & 0.600 & 0.894\\
            (baseline) & 0.756 & 0.675 & 0.860 & 0.763 & 0.685 & 0.862 & 0.753 & 0.653 & 0.889\\
            LSTM & *0.748 & 0.691 & 0.815 & 0.746 & 0.689 & 0.812 & 0.748 & 0.670 & 0.848\\
            (peep-hole) & 0.763 & 0.699 & 0.839 & 0.748 & 0.685 & 0.824 & 0.769 & 0.687 & 0.873\\
            \multirow{2}{2cm}{GRU} & *0.754 & 0.717 & 0.796 & 0.738 & 0.689 & 0.794 & 0.761 & 0.698 & 0.835\\
                                 & 0.762 & 0.708 & 0.825 & 0.758 & 0.707 & 0.816 & 0.764 & 0.687 & 0.861\\
            \multirow{2}{2cm}{CIFG} & *0.742 & 0.664 & 0.839 & 0.749 & 0.662 & 0.863 & 0.738 & 0.640 & 0.872\\
                                  & 0.780 & 0.735 & 0.832 & 0.762 & 0.707 & 0.826 & 0.788 & 0.725 & 0.863\\
        \bottomrule
    \end{tabular}
    \label{tab:tab1}
\end{table}
去除词性影响后..(一顿分析)

可见在两种不同预训练嵌入下,xxx在命名实体识别上取得了更佳的性能。

使用字词向量结合的嵌入模型试验结果
\begin{table}[H]
    \centering
    \caption{实验五:字词向量结合命名实体识别结果}
    \begin{tabular}{cccccccccc}
        \toprule
            \multirow{2}{*}{Cell} &\multicolumn{3}{c}{Overall} &\multicolumn{3}{c}{B} &\multicolumn{3}{c}{I}\\
            \cmidrule(lr){2-4} \cmidrule(lr){5-7} \cmidrule(lr){8-10}
            & F1 & R & P & F1 & R & P & F1 & R & P\\
        \midrule
            LSTM & *0.724 & 0.621 & 0.866 & 0.736 & 0.635 & 0.874 & 0.718 & 0.600 & 0.894\\
            (baseline) & 0.756 & 0.675 & 0.860 & 0.763 & 0.685 & 0.862 & 0.753 & 0.653 & 0.889\\
            LSTM & *0.748 & 0.691 & 0.815 & 0.746 & 0.689 & 0.812 & 0.748 & 0.670 & 0.848\\
            (peep-hole) & 0.763 & 0.699 & 0.839 & 0.748 & 0.685 & 0.824 & 0.769 & 0.687 & 0.873\\
            \multirow{2}{2cm}{GRU} & *0.754 & 0.717 & 0.796 & 0.738 & 0.689 & 0.794 & 0.761 & 0.698 & 0.835\\
                                 & 0.762 & 0.708 & 0.825 & 0.758 & 0.707 & 0.816 & 0.764 & 0.687 & 0.861\\
            \multirow{2}{2cm}{CIFG} & *0.742 & 0.664 & 0.839 & 0.749 & 0.662 & 0.863 & 0.738 & 0.640 & 0.872\\
                                  & 0.780 & 0.735 & 0.832 & 0.762 & 0.707 & 0.826 & 0.788 & 0.725 & 0.863\\
        \bottomrule
    \end{tabular}
    \label{tab:tab1}
\end{table}

综合比较下

\section{本章小结}
扯一扯
