\chapter{绪论}
\label{chap:introduction}
\section{课题背景及意义}
自然语言处理(Natrual Language Processing, NLP)是人工智能研究的一个重要组成部分,其研究内容和方法涉及到包括语言(音)学、计算机科学、数学和认知科学等在内的众多学科,其研究目的在于借助计算机处理、加工人类特有的自然语言信息,乃至生成人类可接受的语言文字信息,实现人机智能交互系统的构建。
早期的自然语言处理研究受限于当时认知科学水平和形式化理论的思潮,在发展的过程中摇摆于理性主义和经验主义之间。
20世纪90年代以来,计算机和网络大规模普及,硬件技术和性能指标都有了极大的提升,这为自然语言处理理论方法的研究提供了良好的契机;
同时,日益丰富的语言文字资源也为自然语言处理技术的发展带来了巨大的驱动力。

目前,自然语言处理的研究和应用所涉及的方向包括但不限于以下几个领域:信息检索,机器翻译,自动文摘,文本分类,问答系统、信息抽取、文本挖掘和语音识别与合成等。
在当前互联网产生的信息爆炸式增长的环境下,在海量信息中获取有价值信息已经成为数据科学的主要研究内容。
自然语言处理正是应对此类问题的有效技术,它在社会生产生活中的应用场景也越来越多样化,其核心技术在完成当前纷繁复杂的自然语言信息处理任务时,也面临着来自性能、效率和成本的挑战。
因此,利用机器学习、深度学习等方法,依托海量的语音、文本数据资源,在低人工干预、时间成本可控的前提下,高效、精确地解决自然语言处理问题,成为当前学术与工业界共同的研究热点。

以数据库存储为主要形式的结构化数据中包含大量由特定对象产生的、存在显著关系的信息,这一部分信息由于具有相对稳定的结构和实体关系,故成为前人进行数据挖掘研究的主要对象。
然而,经过了几十年的发展,互联网上包括自然语言在内的非结构化数据,远远多于结构化数据,并且其总量正在高速增长。这些数据具有数据形式多样、语义内容隐藏和处理手段复杂等特点。
我们可以发现,当前环境下非结构化文本具有如下特点:
\begin{enumerate}
    \item 数据体量巨大:在“用户产生内容”这一流行互联网运行模式的作用下,非结构化文本具有近乎“无限”的数据量。
    \item 难以提取有价值信息:文本结构复杂、语义特征隐晦、语言习惯繁多等等因素,给自然语言理解任务带来了众多挑战。
\end{enumerate}
而自然语言文本往往是非结构化的、实体和关系是混杂的,并且海量的数据往往包含大量的无用信息。如何从大量的原始数据中剔除噪声,获取其中有用的信息,就是自然语言处理需要解决的一个重要问题:信息抽取。
信息抽取需要从文本中获取用户感兴趣的知识内容,这些知识内容由文本中的特定元素构建,表现形式一般为事件或事实消息。
以新闻报道为例,文本中的特定元素即实体,包括时间、地点、人物、事件、原因、过程等。要获取这类信息,首先需要将这些内容从原始文本中辨别出来,再获取它们之间的关系,最终构建成为知识条目,作为提取的信息输出。

事实上,中文自然语言处理的大部分应用都面临着中文分词、词性标注、命名实体识别这类问题,我们称之为序列标注问题,即对观测到的文本序列元素,给出其对应的标签,以实现分词或标注。
解决这些问题往往是进行如信息抽取这类任务之前的基础,它们的结果往往对后续任务的效果有着重大的影响。

\section{国内外研究现状}
\label{sec:current}
\subsection{命名实体识别方法的研究现状}
\subsection{神经网络在命名实体识别上的应用现状}
自20世纪90年代以来,机器学习方法崭露头角,它们在众多领域大显身手,包括计算机图像处理、语音识别、人工智能等。在自然语言处理领域,基于大规模语料库的统计方法已经成为主流,这使得中、西文自然语言处理在数学意义上产生了一定的重合。
随着神经网络的发展和深度学习概念的提出,越来越多新的方法和工具涌现出来,在面对复杂多样的自然语言处理任务时,我们获得了更多应对从数据质、量不足到模型训练、优化等问题的手段,从而能够更好地解决各种自然语言处理问题。
\subsection{中文命名实体识别的难点}
相比于以英文、法文为代表的西方语言命名实体识别,中文命名实体识别研究受限于语言本身的特性,存在更为棘手的困难,最主要的就是与分词有关的词边界确定问题。
汉语以字为基本单位的语言特征同大部分西文以词为基本单位的特征存在差异。
西方语言以屈折语为主。
屈折语文本中词与词之间带空格等明显表示单词界限的标志,同时还存在性、数、格和字母大小写等比词粒度更小的词缀特征。
而以汉语为代表的(还包括日语、藏语等)亚洲语言属于分析语或黏着语,它们通过词序与虚词的增减表达语法意义,即不存在词的格之分;
或在词根前中后黏贴不同词尾表达语法意义。
同时,就分词技术而言,也存在汉语分词规范不统一、歧义切分和未登录词处理等问题,分词结果同命名实体识别结果之间存在复杂的相互作用。
这在很大程度上降低了语言的结构特征,提高了命名实体识别的难度。
\section{主要研究内容与文章结构}
本文在总结现有传统的基于统计机器学习的命名实体识别方法的基础上,探讨了基于神经网络的方法在中文命名实体识别任务上的应用,并针对两种不同的任务场景,分别提出了改进方法。
主要工作描述如下:
\begin{enumerate}
    \item 总结常见的基于统计的命名实体识别方法,分析比较不同方法之间的原理差异,详细介绍了基于神经网络的命名实体识别方法,指出了其相较于传统方法之间的优势
    \item 针对基于神经网络的命名实体识别方法在文言文语料上的应用进行了探讨,在给定的应用场景下进行了实验,并给出了优化方法
    \item 分别探讨了使用字、词向量进行命名实体识别的局限性,并提出了结合字词向量进行命名实体识别的方法
    \item 提出一个针对文中使用的命名实体识别架构中神经网络输出层的改进策略,并分析了该策略对命名实体识别结果的潜在影响
    \item 通过实验证明了文中提出的各改进方法的有效性,并分析方法可用的场景
\end{enumerate}
本文各章的内容安排如下:

第一章介绍了本文的选题背景和研究意义,命名实体识别与自然语言处理的关系、研究现状和文章内容结构。

第二章介绍了传统的基于统计的方法的技术原理,分析比较它们在命名实体识别任务上的性能和优缺点。

第三章介绍了基于神经网络的序列标注框架及其各部分的原理,包括字、词嵌入,循环神经网络与长短时记忆网络,在中医领域的医案文言文本上进行了命名实体识别实验,并给出了优化方法。

第四章针对一般语料上的命名实体识别任务,提出了结合字词向量的命名实体识别方法和建立全文字典转移矩阵的改进策略,分析了两种改进方案的原理和作用,并通过实验验证其有效性。

第五章对全文工作进行了总结,针对现有方法与文中所做研究的局限性与不足之处,指出了下一步的工作方向。
