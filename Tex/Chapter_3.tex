\chapter{基于LSTM-CRF的中文序列标注模型}
\section{引言}
由于基于统计的方法在总是在一定的假设前提下对自然语言进行建模,其复杂程度远远不能与灵活多变的自然语言相比,且大多数基于统计的模型若想取得较好的识别效果,总要较大程度地依赖人工构建语料库、字典、特征等手段,且不同语言、不同领域语料之间模型的可迁移性也较差,这在一定程度上限制了基于统计的模型在自然语言处理中进一步的应用。

早在上世纪80年代,神经网络就作为机器学习领域的一个重要的研究方向受到关注。
人工神经网络可以定义为”由具有适应性的简单单元组成的广泛并行互联的网络,它的组织能够模拟生物神经系统对真实世界的物体作出的交互反应“。
其中,这个”具有适应性的简单单元“来源于1943年McCulloch和Pitts提出的M-P神经元模型。
这一模型实际上就是对生物学意义上的神经元进行的一种形式化的数学描述。
早期的神经网络结构较为简单,受到硬件计算能力的限制,一般层数较少,多用于简单模型中;由于结构复杂的网络在训练中存在严重的过拟合问题,且训练效率很低,导致神经网络的发展陷入低谷。
Hinton在2006年提出深度学习的概念,指出深层神经网络可以通过组合低层特征获得抽象的高层特征,这些特征能够更好地对数据的信息进行整合抽象。
同时还提出了基于深度置信网络的非监督贪心逐层训练算法,能够较好地适应深层网络结构,解决了当网络层数较多时的训练难的问题。

深度学习方法在多个领域都有着重要的应用。
Lecun等人提出的卷积神经网络(convolutional neural network,CNN)在图像识别领域得到了广泛应用。
A.Krizhevsky和Hinton等人在2012年将CNN应用于ImageNet数据集上,取得了图像分类和目标定位的最佳结果。
此后CNN在各项任务中都大显身手,在人脸识别、视频分类、行为识别等方向上都取得了较好的效果。
而对于以序列信号作为输入的问题,如语音识别、机器翻译和序列标注等方向,循环神经网络(recurrent neural network,RNN)则成为了解决这类问题的主要手段。
这其中又以能够较好地获取较长距离内上下文依赖的长短时记忆网络(long short-time memory, LSTM)最为突出,

本章针对命名实体识别问题,从几个方面对实现实体识别的基于LSTM-CRF的中文序列标注框架进行介绍。
本章将对词的分布式表示进行介绍,考察循环神经网络的基本结构,对比分析长短时记忆网络及其各种变体的实现,最后在一个具体的命名实体识别任务场景下,讨论基于LSTM-CRF的序列标注框架,并给出针对该任务基于命名实体的额外字符级特征的改进方法,最后设计实验验证改进方法的有效性。

\section{词的分布式表示}
\subsection{从统计语言模型到神经网络语言模型}
不论是传统的基于统计的方法还是基于神经网络的方法,进行命名实体识别任务首先要将自然语言词汇转化为模型能够处理的形式,即实值向量。
最基本的词的向量化表示就是One-hot Representation,其形式为一个维度为词表大小的$\{0, 1\}$向量,设该词在词表中的索引为$i$,则其one-hot向量只在位置$i$上的值为1,其余位置为0。
这种表示方法有很多缺陷,首先不同的词并不能从其对应的one-hot向量中判别其语义相关程度,它基本不包含词汇的语义信息;其次,获取one-hot向量的方式仅仅是通过建立词表即可,忽略了词汇的上下文特征而完全将词独立对待;最后,这种向量形式显然非常稀疏,在深度神经网络模型中非常容易导致维数灾难。

词的分布式表示最初由Hinton于1986年提出。
与one-hot Representation不同的是,分布式表示下的词向量维度不再是词表大小,而是一个相对较低的固定维度。其中的每一个分量也不再是0或者1,而是一个实值。
这时词汇表可以视作是一个向量空间,词即是空间中的点。
既然是空间,就可以定义距离,这个距离可以理解为词汇在语义或语法层面上的距离。
这样词与词之间就不是孤立的了,词在上下文中的语义信息也得到了保留,词向量维度人为可控且远远小于词表大小。

估计词向量参数的方法有很多种,包括隐含语义分析(latent semantic analysis, LSA)、隐狄利克雷分布(latent Dirichlet allocation)和利用神经网络的方法。

Bengio等人在2003年提出了基于神经概率的语言模型。利用神经网络估计语言模型参数的同时,得到了作为副产品的词向量。
其基本的思路是,首先赋予每个词一个词特征向量$\Vector{v}(\Vector{v}\in \Set{R}^m)$,$m$为可以设定的词向量维度,如50,100,200等;然后使用这些词特征向量来表示一段文本序列的联合概率函数:
最后在训练过程中,同时更新(学习)词特征向量和概率函数的参数(语言模型参数)。

在章节2.1.1中介绍了n元语言模型。语言模型的目标是计算给定序列在该语言下的概率,这个概率通过一系列条件概率的乘积表示:
\begin{equation}
    P(w^T_1) = \prod^{T}_{t=1}P(w_t|w_1^{t-1})
\end{equation}
其中$T$为序列长度,$w_i^j = (w_i, w_{i+1},\dots, w_j)$。但由于这种计算方式参数过多,后来模型可以简化为序列某时刻词的条件概率只依赖其上文的若干个词(n-gram):
\begin{equation}
    P(w_t|w_1^{t-1}) \approx P(w_t|w^{t-1}_{t-n+1})
\end{equation}
通过统计各词序列出现的次数作为条件概率值,填充训练模型参数。

另一种估计模型参数的方法是对模型建立一个目标函数,将参数训练转化为对一个关于模型参数和训练数据的函数优化问题。
实际中,常使用最大对数似然函数作为目标函数,通过调整其参数获得对数似然函数的最大值,此时得到的就是较为理想的参数。
\begin{equation}
    \mathcal{L} = \sum_{c\in C} \log p(w|Context(w))
\end{equation}
在语言模型中,关于某时刻词$w$上下文的条件概率可以用$Context(w)$表示。于是其中的条件概率就可以表示为关于词和其上下文的$Context(w)$的函数:
\begin{equation}
    p(w|Context(w)) = F(w, Context(w), \theta)
\end{equation}
其中$\theta$就是待训练参数集。

在神经语言模型中,获取关于词以其上文为条件的条件概率
\begin{equation}
    f(w_t, \dots, w_{t-n+1}) = P(w_t|w_1^{t-1})
\end{equation}
被分解为两部分,其一是将词表$V$中的词映射为实值向量的矩阵$\Matrix{C}$,其维度是$|V|\times m$;
其二是获得条件概率的映射函数$g$,该函数接受序列$w_{t-n+1}^{t-1}$在实值矩阵$C$中的对应各行向量作为输入,输出给定序列下$w_t$是词典中某个词的概率分布。
于是上面的函数$f$可以进一步表示为:
\begin{equation}
    f(i, w_{t-1}, \dots, w_{t-n+1}) = g(i, C(w_{t-1}), \dots, C(w_{t-n+1}))
\end{equation}

在具体的操作上,模型采用了一个三层神经网络进行参数学习,结构包括输入层、隐藏层和输出层。
值得注意的是其输入层进行的操作是将词$w$上文$n-1$个词的词向量首尾相连得到特征向量$x_w$输入到隐层中。
网络使用传统的前后向传播方法进行参数学习,与其他模型不同的是,在后向传播的最后一步,通过计算$\frac{\partial\mathcal{L}}{\partial x}$并取其中相应维度的部分,实现了一次训练之后批量更新作为输入的$n-1$个词向量的过程。

\subsection{Word2Vector基本原理}
在Bengio工作的基础上,谷歌的Tomas Mikolov团队实现了开源工具word2vec,其中使用了两个模型:CBOW(continous bag of word)模型和Skip-gram模型。

与之前提到的根据上文预测下一个词不同的是,CBOW通过设置一个固定大小的上下文窗口$c$,如设置为2,利用$w_{t-2}, w_{t-1}, w_{t+1}, w_{t+2}$预测当前词$w_t$,Skip-gram则是通过当前词$w_t$预测其上下文。
神经网络模型将上文词向量连接作为隐层输入,隐层中的激活函数为$tanh$。CBOW同样使用三层的神经网络结构,包括输入层、投影层和输出层,但投影层的输出是通过窗口内$2c$个词向量的和构建的。
并且CBOW模型中并不存在隐藏层,也就没有激活函数。
最后,CBOW对输出层进行了较大的改造:通过Huffman树来构造输出层,将求解词关于其上下文的条件概率的问题转化为二分类问题。
具体来说,输出层构建了一颗包含词典中所有词、并且以这些词为叶子结点的Huffman树。则对于词典中任意一个词,有且仅有一条从根结点出发的路径到达该叶子结点。
从根结点出发,到达每一个非叶子结点时,都会发生一次二分类。假设给定分类输入$\Vector{x_w}$,对任意一次二分类,分类到右分支称为正类,记作$d^w_i=1$,分类到左分支称为负类,记作$d^w_i=0$,则分类概率分别可定义为:
\begin{align}
    p_{right} &= \sigma(x_w^\mathrm{T}\theta) = \dfrac{1}{1+e^{-x^\mathrm{T}\theta}}
    p_{left} &= 1 - p_{right}
\end{align}
其中$\theta$就是需要训练的参数。于是,对于一个给定的词$w$,确定其关于上下文的条件概率的过程,就是根据该词的Huffman编码,得到其路径上每一次二分类分到该类$d^w_i$的概率,并将这些概率乘积:
\begin{equation}
    p(w|Context{w}) = \prod^{l(w)}_{j=2}p(d^w_j|\Vector{x_w}, \theta^w_{j-1})
\end{equation}
其中$l(w)$是$w$所在结点的深度。最后通过构建目标函数
\begin{equation}
    \mathcal{L}(w,j) = (1-d^w_j)\log[\sigma(\Vector{X_w}^\mathrm{T}\theta^w_{j-1})] + d^w_j\log[1-\sigma (\Vector{x_w^\mathrm{T}}\theta^w_{j-1})]
\end{equation}
并调整参数使函数值最大化来训练模型。

当确定了$\theta$之后,模型也就确定了。同时在每一次更新参数的过程中,词$w$上下文窗口内的词向量也得到了更新:
\begin{equation}
    \Vector{v(\tilde{w})} := \Vector{v(\tilde{w})} + \eta\sum^{l{w}}_{j=2}\dfrac{\partial\mathcal{L}(w,j)}{\partial \Vector(x_w)}
\end{equation}
其中$\eta$为$w$对上下文词向量贡献的权重系数。

与CBOW相对的Skip-gram模型则在训练过程上十分类似,只是其模型投影层不再有加和等操作,而只把输入层的结果直接用来构造Huffman树。最后训练的过程则是求$p(Context(w)|w)$。具体就不再赘述。

上述过程称为基于Hierachical Softmax的模型。为了简化训练过程,提高效率并且改善词向量的质量,Mikolov等人又采用Negative Sampling来对模型进行优化,避免了构建较为复杂的Huffman树。
以CBOW模型为例,其思想是指定一个关于当前词$w$的负样本集$NEG(w)$,采取下面的定义来获得词的正负类别标签,用来取代通过Huffam树获得的分类标签:
\begin{equation}
    L^w(\tilde{w}) = \left\{
        \begin{aligned}
            1, &\tilde{w} = w;
            0, &\tilde{w} \neq w;
        \end{aligned}
    \right.
\end{equation}
则给定训练样本$(Context(x), x)$,通过最大化
\begin{equation}
    g(w) = \prod_{u\in\{w\}\cup NEG{w}}p(u|Context(w))
\end{equation}
推导得到目标函数
\begin{equation}
    \mathcal{L} = \sum_{w\in C}\{\log[\sigma(\Vector{x_w}^\mathrm{T}\theta^w)] + \sum_{u\in NEG(w)}\log[\sigma(-\Vector{x_w}^\mathrm{T}\theta^u)\}
\end{equation}
用来训练模型参数。

\section{循环神经网络与长短时记忆网络}

\subsection{循环神经网络}
\subsection{长短时记忆网络及其变体}
虽然除$t_0$时刻的RNN单元外后续的每个单元接收的输入都带有之前所有时刻信息的累积,但受限于梯度消失问题,相隔较远的时刻输入对参数学习将不会有贡献,导致时域上靠后的单元无法很好地感知靠前的单元,句子内的长程依赖信息就无法被学习。针对这一问题,S.Hochreiter和J.Schmidhuber在1997年提出了LSTM,其本质是使用带有信息存储和更新机制的单元代替RNN中一般的激活单元。信息存储即cell的内部状态,而更新机制则通过一系列“门”实现,包括输入门$i_t$、遗忘门$f_t$和输出门$o_t$。
一个基本的LSTM实现如图所示。其内部参数可表达为:
\begin{align*}
    \Vector{i_t} &= \sigma(\Matrix{W_{hi}}\Vector{h_{t-1}} + \Matrix{W_{xi}}\Vector{x_{t}} + \Vector{b_i})\\
    \Vector{f_t} &= \sigma(\Matrix{W_{hf}}\Vector{h_{t-1}} + \Matrix{W_{xf}}\Vector{x_{t}} + \Vector{b_f})\\
    \Vector{o_t} &= \sigma(\Matrix{W_{ho}}\Vector{h_{t-1}} + \Matrix{W_{xo}}\Vector{x_{t}} + \Vector{b_o})\\
    \Vector{C_t} &= \Vector{f_t}\odot\Vector{C_{t-1}} + \Vector{i_t}\odot tanh(\Matrix{W_{Ch}}\Vector{h_{t-1}} + \Matrix{W_{Cx}}\Vector{x_t} + \Vector{b_{C}})\\
    \Vector{h_t} &= \Vector{o_t} \odot tanh(\Vector{C_t})
\end{align*}
其中$\odot$代表元素乘积,其结果仍是一个高维实向量。
根据该描述可知,对于某时刻的输入,LSTM层在得到输出之前,
首先对cell中存储的历史信息$C_{t-1}$进行了处理,通过遗忘门层$f_t$根据上一时刻隐层输出$h_{t-1}$与当前时刻的输入$x_t$,
控制cell状态中哪些信息将得到保留;
随后再将更新后的cell状态的激活值与输出门层$o_t$结合得到当前时刻的隐层状态传递下去。
文献[]给出了LSTM cell的一个改进版本,它在原有LSTM结构的基础上增加了cell状态对遗忘信息的影响,
称为peep-hole,即在图[]所示LSTM结构基础上增加虚线部分。
其内部参数可表示为:
\begin{align*}
    \Vector{i_t} &= \sigma(\Matrix{W_{hi}}\Vector{h_{t-1}} + \Matrix{W_{xi}}\Vector{x_{t}} + \Matrix{W_{Ci}}\Vector{C_{t-1}} + \Vector{b_i})\\
    \Vector{f_t} &= \sigma(\Matrix{W_{hf}}\Vector{h_{t-1}} + \Matrix{W_{xf}}\Vector{x_{t}} + \Matrix{W_{Cf}}\Vector{C_{t-1}} + \Vector{b_f})\\
    \Vector{o_t} &= \sigma(\Matrix{W_{ho}}\Vector{h_{t-1}} + \Matrix{W_{xo}}\Vector{x_{t}} + \Matrix{W_{Co}}\Vector{C_{t-1}} + \Vector{b_o})\\
    \Vector{C_t} &= \Vector{f_t}\odot\Vector{C_{t-1}} + \Vector{i_t}\odot tanh(\Matrix{W_{Ch}}\Vector{h_{t-1}} + \Matrix{W_{Cx}}\Vector{x_t} + \Vector{b_{C}})\\
    \Vector{h_t} &= \Vector{o_t} \odot tanh(\Vector{C_t})
\end{align*}
在上述使用peep-hole的LSTM cell的基础上,
文献[]在命名实体识别任务上还使用了耦合输入-遗忘门(Coupled Input-Forget Gate, CIFG)[]取得了较好的效果,
即令$\Vector{f_t} = 1 - \Vector{i_t}$,对比LSTM cell简化了CIFG的计算:
\begin{equation*}
    \Vector{C_t} = (1 - \Vector{i_t})\odot \Vector{C_{t-1}} + \Vector{i_t}\odot tanh(\Matrix{W_{Cx}}\Vector{x_t} + \Matrix{W_{ch}}\Vector{h_{t-1}} + \Vector{b_C})
\end{equation*}
最后,文献[]提出了一种更为简单的LSTM单元的变体,称为门控循环单元(Gated Recurrent Unit,GRU)。
GRU存在多种变体,区别在于计算每个门层输出是否使用之前的隐藏状态和偏置。
本节使用的GRU仅使用隐藏状态计算门层输出,如图3所示。
其最主要的特征是
\begin{enumerate}
    \item 融合了cell状态和隐藏层输出
    \item 融合了输入门和遗忘门
\end{enumerate}
这使得单元相比LSTM更加简单,参数更少,同时实验表明它能够在大部分任务中取得与LSTM相当或更佳的效果。
其内部参数可表示为:
\begin{align*}
    \Vector{z_t} &= \sigma(\Matrix{W_{zx}}\Vector{x_t} + \Matrix{W_{zh}}\Vector{h_{t-1}})\\
    \Vector{r_t} &= \sigma(\Matrix{W_{rx}}\Vector{x_t} + \Matrix{W_{rh}}\Vector{h_{t-1}})\\
    \Vector{\tilde{h_t}} &= tanh(\Matrix{W_{\tilde{h}u}}(\Vector{r_t} \odot \Vector{h_{t-1}}) + \Matrix{W_{\tilde{h}x}}\Vector{x_t})\\
    \Vector{h_t} &= (1 - \Vector{z_t}) \odot \Vector{h_{t-1}} + \Vector{z_t} \odot \Vector{\tilde{h_t}}
\end{align*}

\section{基于Bi-LSTM-CRF的医案症状术语识别}
\subsection{任务背景}
中医典籍和医案是中国传统医学宝贵的知识财富。
中医医案经历了长时间的积累,记载了古今众多医者的实证经验和临床诊疗规律。
这些医案中存在大量的症状信息,这些信息对医者辩证论治、察明证素,进而对证施治起着至关重要的作用;
同时,症状、治法、处方、剂量与预后疗效之间存在的内在联系,也能够反映中医对疾病的诊疗机理。
若要系统地研究这些诊疗机理,对医案进行结构化处理,挖掘医案文本,寻找其中蕴含的关联规则是一系列必要的任务。
在医案结构化处理任务中,首要的就是将中医医案文本中对症状具体的描述和长期以来形成的普遍使用的中医症状术语提取出来。

目前,针对中医医案文本中症状术语的识别存在几个难点。
首先,中医医案的结构化程度低。
具体表现为不同年代、不同地域的不同医家个体,其记录医案的结构和习惯均存在不同,导致无法按统一标准进行结构化。
其次,就医案内容而言,由于存在大量医案年代较为久远,其语言特点与现代汉语也存在较大差异,多为文言文或半文半白,识别难度较大。
最后,对于需要提取的目标症状描述或症状术语,缺乏统一的规范。
至今,中医临床常见症状术语尚未有国家标准,中医医案中口语化、症状实体包含、互指现象也十分普遍,加之症状描述本身形式灵活多变,这对识别症状的正确性、完整性提出了更高的要求。

识别中医医案文本中的症状描述或症状术语这一任务实际上就是从医案文本中进行症状实体识别,可以将其视为命名实体识别任务的一种。
目前对于中医医案症状识别主要使用条件随机场(Conditional Random Field, CRF)方法。
文献[]从命名实体识别的方法出发,对比了CRF与支持向量机、最大熵模型等常见的命名实体识别方法,指出了CRF在该任务上的有效性,但同时也存在依赖分词和词性标注效果、对超过五个字符的较长实体识别效果较差等局限性。
文献[]对比了CRF、隐马尔科夫模型和最大熵隐马模型在特定病种的标准病历数据集上对症状、疾病和诱因的识别效果。
文献[]-[]均采用了CRF方法针对中医领域的文本语料进行命名实体识别,以进行医案结构化、症状等中医术语提取等工作。

本节针对CRF方法特征选择相对复杂,人工标注特征代价相对较高,对语料结构化、标准化程度依赖较大等局限性,提出使用结合神经网络的序列标注框架,
在以文言文本为主的近代中医医案典籍上进行中医症状术语识别,并根据症状组成要素,添加简单的字符级别特征,在增加较少人工标注成本的前提下,获得更佳的症状识别效果。
本节将使用不同的LSTM结构进行实验,并对比分析不同结构对实验结果的影响。

\subsection{双向LSTM-CRF序列标注框架}
单向LSTM能够较好地获取上文信息,使得句子中的历史依赖特征可以保留下来。
对于一个可能含有症状术语的文本序列$\Tensor{X}$,其$t$时刻的输入为$\Vector{x_t}$,$x_t$为文本序列中单个文字对应的特征向量。
在未加入症状字特征时,该特征向量为该文字对应的字向量。
将字向量序列按顺序输入,则LSTM层在$t$时刻的隐层输出$\Vector{\overrightarrow{h_t}}$代表根据包含了上文特征语义信息和当前时刻输入信息得到的字向量表示。
但传统中医医案中对于症状的描述出现的范围包括证候、诊断(包括复诊、三诊等)、预后疗效、病案分析。
判定文本序列的某个子序列是典型的症状术语,需要同时考虑上下文携带的信息。
同理,使用另一个结构相同但参数独立的LSTM层,反向输入给定序列则可得到根据后文特征得到的字向量表示$\Vector{\overleftarrow{h_t}}$。
将这两部分字向量表示拼接可以得到包含上下文信息的字向量表示$\Vector{h_t} = [\Vector{\overrightarrow{h_t}}, \Vector{\overleftarrow{h_t}}]$。

双向LSTM-CRF序列标注框架中,Bi-LSTM层用来整合提取输入序列的特征,将字向量结合上下文语义信息扩展为向量表示,并在其输出层采用Softmax函数将隐层输出映射到标签集上的概率分布向量。
文本序列输入完毕后,LSTM层的最终输出为序列中每个字符所述标签的概率分布矩阵。最后,使用CRF层根据LSTM层得到的概率分布矩阵,在所有可行的标签序列空间中确定一个最合理的序列路径,此时序列中每个标签对应的即是相同位置上的字符标签。

\subsection{基于症状要素的字特征}
考虑到中医症状实体本身还具有很多明显的特征,以这些特征作为锚点,能够快速准确地定位一些出现比较频繁的症状。
但同时也存在小部分负例,应以文本序列出现的上下文信息判断是否是症状。
为了考察这些特征对症状识别结果的影响,本节进行了如表1所示下的特征标签分类。
对于文本序列中的单字,模型使用预训练的字嵌入作为基本特征;额外的症状术语单字,将采用随机初始化获得其表示其标签类型的嵌入向量,与字嵌入连接作为该字符的表示。
表1所示特征,仅为对标准化的症状词典进行简单的字符频率统计并按照字符表意类型进行分类得到的常见额外字特征。
\begin{table}[!htbp]
    \centering
    \footnotesize
    \setlength{\tabcolsep}{4pt}
    \renewcommand{\arraystretch}{1.2}
    \begin{tabular}{cp{7cm}c}
        \hline\hline
        特征种类 & 部分特征关键字 & 特征标签\\
        \hline
        自感特征 & 疼、痛、肿、胀、酸、麻 & SF(self feeling)\\
        \hline
        身体部位 & 脉、左、右、寸、关、尺、舌、苔、手、身、口、目、眼、耳、足、胸、额、头、脘、腹、唇、齿、胃、面、颊、肢 & BP(body parts)\\
        \hline
        颜色形态 & 白、赤、红、滑、腻、溏、稀、干、燥、朱、黑、青、紫、绛、焦、黄 & CS(color shape)\\
        \hline
        程度描述 & 稍、微、略、甚、强、大、多、少、无、极、弱 & AD(adverbs of degree) \\
        \hline
        排泄物名词 & 便、溲、痰、汗 & EC(ecreta)\\
        \hline\hline
    \end{tabular}
    \caption{症状字特征分类}
\end{table}

上述特征均是字符级别。
为了不增加训练语料的人工标注成本,可以在训练语料中进行简单的单字替换获得,作为单独的特征列输入模型。
最终的标注结果示例如图5所示。

本节将单字特征和预训练的字嵌入进行连接获得的新的字向量表示作为模型输入,这些额外特征将作为字嵌入的扩展,为模型提取特征从而进行症状术语识别提供更具体有效的信息。


\subsection{实验设计}
本节使用的数据集来自中医典籍《全国名医验案类方》,作者为近代名医何廉臣,共计492例医案。
原医案语料已具有一定的结构特征,单个医案中划分了病因、证候、诊断、疗法等不同项目,但也存在包括医者、病者等无关信息,本节只选取其中证候、诊断、疗法、效果及作为医案评述的“廉按”部分内容作为原始语料,共2959条语料。
由于典籍中的医案是医者行医记录,虽然在医者已按照既定结构记载,但这些项目中仍存在大量描述和推断,不能完全作为症状实体。
且记录医案存在一定随意性,症状实体在证候、诊断、疗法、效果和医案评述中均可能出现,故对所有语料参考《中医临床症状术语规范》中六大类37小类共2069条规范症状进行标注。
考虑到原始语料年代较为久远,部分症状描述可能与现代汉语存在差异,故酌情对语料中频繁出现的部分症状进行了补充标注。
需要说明的是,文献[1]和[4]的工作已经表明,对于中医医案尤其是使用古代汉语的中医医案语料,对其进行分词效果并不好,因为现有的分词工具多倾向于将无法识别的词语进行单字切分,若以词为粒度进行识别,分词会对结果产生负面影响。
因此本节利用字符级别特征进行序列标注,对所有语料进行单字切分,并按照BIO标准进行标注,即症状实体起始字标签为“B”,症状实体非起始字为“I”,非症状实体字为“O”。
由于实验语料规模较小,故本节选取其中30例医案作为测试语料,剩余语料作为训练数据,其中训练集与验证集比例为9:1。
最后,本节使用Gensim工具利用全部中医医案语料作为训练数据,获得了预训练的字嵌入作为基本特征,字嵌入维度设置为100,基于症状要素的字特征嵌入维度设置为30。
本节基于是否加入症状字特征和多种LSTM cell实现进行对比实验。实验结果基于标签进行评估。

本节使用隐藏层单元数量为64的双向LSTM,训练语料句子长度最大不超过100个字符。
为防止过拟合,建立模型时对SLTM的输入层和隐藏层进行随机dropout,保留概率均设置为0.8。
同时,全连接层L2正则化参数为0,001,全局学习率为0.002。

实验结果表明,基于LSTM-CRF的序列标注模型在中医医案症状识别任务中F1值最好达到了0.78。
在对中医医案进行序列标注的任务中,提高召回率是一个难点,这与文献[]在相似语料上的实验得出的结论一致。
在未加入症状字特征的条件下,得益于相对简单的结构和较少的待训练参数,GRU取得了相对较好的识别效果。
增加症状字特征后,可以看出不同的LSTM实现在标签识别的效果上均有不同程度的提高,说明常见症状字特征对识别结果有一定的积极影响。

另外,CIFG在加入症状字特征后,各标签的召回率和精确率都得到了提高。
也可以看出,与基线实验相比,LSTM(with peep-hole)、GRU和CIFG三种带有peep-hole实现的单元结构对增加的特征更加敏感,特征对提升它们在症状实体识别任务上的表现具有更明显的作用。
同时,使用LSTM-CRF序列标注模型相比单纯的CRF能够更好地识别较长的症状术语,改善了CRF对长症状术语不能很好召回的情况[]。
各方法对测试语料中五字以上长症状术语的识别效果如表3所示。


\section{本章小结}

本章在基于LSTM-CRF序列标注模型的基础上,针对中医症状特征添加了部分字符级别的要素特征,在小规模以古汉语为主的中医医案语料上的实验表明,该模型泛化能力更强,且对较长实体的识别效果更好。
同时,由于中医症状种类繁多,加之古汉语精干灵活的语言特点,该方法也存在召回率较低的缺点。
因此后续工作将围绕以下几点展开:第一,比较多种不同的标签模式对症状术语识别的效果影响;第二,由于部分症状描述分散于多个短语中,并且存在大量比较(如“六脉俱缓,左关尤甚”)、持续(如“继则全体大热,昼夜不休”)等特征,寻找有效方法利用这些特征也至关重要;最后还需考虑参照症状字典对不标准的症状描述进行聚类,识别时按照标准症状建立自然语言序列与症状的映射关系,便于以后对古医案的结构化。
