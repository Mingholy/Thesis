\chapter{作者简历及攻读学位期间发表的学术论文与研究成果}

\section*{已发表(或正式接受)的学术论文:}

[1] 李明浩, 刘忠, 姚远哲. 基于LSTM-CRF的中医医案症状术语识别[J]. 计算机应用, 录用.

\chapter{致\quad 谢}
时光飞逝,转眼间三年的研究生生涯即将告一段落。回顾三年的求学之路,无论是在学业还是生活上,自己都成长颇多。在论文即将落笔之际,谨向在研究生阶段给予我帮助、支持与陪伴的老师和同学表示诚挚的感谢。

首先,衷心地感谢我的导师刘忠教授,感谢刘老师在我研究生学习中每一个重要阶段的悉心指导和鼓励。在我遇到困难与疑惑时,您能为我提出中肯的建议和意见;在我渴望更大的学习空间时,您为我积极提供良好的学习资源和空间,使我的学术视野得到了开阔,在研究和工程能力上都得到了锻炼和提升。刘老师对学术研究的严谨求实、对问题透彻的理解以及高尚的师德和高度敬业精神都使我受益匪浅。

其次,衷心感谢电子科技大学的姚远哲老师。感谢姚老师在课题研究中给予我大力的支持和帮助,在生活中给予我的热情关怀和照顾。同时,还要感谢中国科学院成都计算机应用研究所、电子科技大学自动推理实验室、乐山职业技术学院和知道创宇成都IA实验室的每一位给予我帮助和支持的老师、同学和同事,在这里向你们表示感谢。

再次,感谢父母在我求学生涯中给我的支持和鼓励,你们是我不断进取的动力,是我勇敢前行的坚实后盾。还要感谢马清山、张佳旭以及各位同窗的陪伴和支持,感谢陪伴我走过硕士研究生阶段的学习和生活,一同分担了很多喜悦和压力。感谢侯雪婷曾经给予的鼓励、支持和陪伴。

最后,感谢中国科学院大学为我们提供的优越的学习资源,感谢电子科技大学数学科学学院在本科期间赋予我的宝贵知识财富。感谢各位论文评审、评议和答辩委员会的专家百忙之中对我工作的指导和建议,您们辛苦了!

落笔至此,愿下阶段人生顺利起航!